\documentclass[]{article}
\usepackage{lmodern}
\usepackage{amssymb,amsmath}
\usepackage{ifxetex,ifluatex}
\usepackage{fixltx2e} % provides \textsubscript
\ifnum 0\ifxetex 1\fi\ifluatex 1\fi=0 % if pdftex
  \usepackage[T1]{fontenc}
  \usepackage[utf8]{inputenc}
\else % if luatex or xelatex
  \ifxetex
    \usepackage{mathspec}
  \else
    \usepackage{fontspec}
  \fi
  \defaultfontfeatures{Ligatures=TeX,Scale=MatchLowercase}
\fi
% use upquote if available, for straight quotes in verbatim environments
\IfFileExists{upquote.sty}{\usepackage{upquote}}{}
% use microtype if available
\IfFileExists{microtype.sty}{%
\usepackage{microtype}
\UseMicrotypeSet[protrusion]{basicmath} % disable protrusion for tt fonts
}{}
\usepackage[right=2.5in]{geometry}
\usepackage{hyperref}
\hypersetup{unicode=true,
            pdftitle={Introduction to R},
            pdfauthor={Abdirisak Mohamed},
            pdfborder={0 0 0},
            breaklinks=true}
\urlstyle{same}  % don't use monospace font for urls
\usepackage{color}
\usepackage{fancyvrb}
\newcommand{\VerbBar}{|}
\newcommand{\VERB}{\Verb[commandchars=\\\{\}]}
\DefineVerbatimEnvironment{Highlighting}{Verbatim}{commandchars=\\\{\}}
% Add ',fontsize=\small' for more characters per line
\usepackage{framed}
\definecolor{shadecolor}{RGB}{248,248,248}
\newenvironment{Shaded}{\begin{snugshade}}{\end{snugshade}}
\newcommand{\KeywordTok}[1]{\textcolor[rgb]{0.13,0.29,0.53}{\textbf{#1}}}
\newcommand{\DataTypeTok}[1]{\textcolor[rgb]{0.13,0.29,0.53}{#1}}
\newcommand{\DecValTok}[1]{\textcolor[rgb]{0.00,0.00,0.81}{#1}}
\newcommand{\BaseNTok}[1]{\textcolor[rgb]{0.00,0.00,0.81}{#1}}
\newcommand{\FloatTok}[1]{\textcolor[rgb]{0.00,0.00,0.81}{#1}}
\newcommand{\ConstantTok}[1]{\textcolor[rgb]{0.00,0.00,0.00}{#1}}
\newcommand{\CharTok}[1]{\textcolor[rgb]{0.31,0.60,0.02}{#1}}
\newcommand{\SpecialCharTok}[1]{\textcolor[rgb]{0.00,0.00,0.00}{#1}}
\newcommand{\StringTok}[1]{\textcolor[rgb]{0.31,0.60,0.02}{#1}}
\newcommand{\VerbatimStringTok}[1]{\textcolor[rgb]{0.31,0.60,0.02}{#1}}
\newcommand{\SpecialStringTok}[1]{\textcolor[rgb]{0.31,0.60,0.02}{#1}}
\newcommand{\ImportTok}[1]{#1}
\newcommand{\CommentTok}[1]{\textcolor[rgb]{0.56,0.35,0.01}{\textit{#1}}}
\newcommand{\DocumentationTok}[1]{\textcolor[rgb]{0.56,0.35,0.01}{\textbf{\textit{#1}}}}
\newcommand{\AnnotationTok}[1]{\textcolor[rgb]{0.56,0.35,0.01}{\textbf{\textit{#1}}}}
\newcommand{\CommentVarTok}[1]{\textcolor[rgb]{0.56,0.35,0.01}{\textbf{\textit{#1}}}}
\newcommand{\OtherTok}[1]{\textcolor[rgb]{0.56,0.35,0.01}{#1}}
\newcommand{\FunctionTok}[1]{\textcolor[rgb]{0.00,0.00,0.00}{#1}}
\newcommand{\VariableTok}[1]{\textcolor[rgb]{0.00,0.00,0.00}{#1}}
\newcommand{\ControlFlowTok}[1]{\textcolor[rgb]{0.13,0.29,0.53}{\textbf{#1}}}
\newcommand{\OperatorTok}[1]{\textcolor[rgb]{0.81,0.36,0.00}{\textbf{#1}}}
\newcommand{\BuiltInTok}[1]{#1}
\newcommand{\ExtensionTok}[1]{#1}
\newcommand{\PreprocessorTok}[1]{\textcolor[rgb]{0.56,0.35,0.01}{\textit{#1}}}
\newcommand{\AttributeTok}[1]{\textcolor[rgb]{0.77,0.63,0.00}{#1}}
\newcommand{\RegionMarkerTok}[1]{#1}
\newcommand{\InformationTok}[1]{\textcolor[rgb]{0.56,0.35,0.01}{\textbf{\textit{#1}}}}
\newcommand{\WarningTok}[1]{\textcolor[rgb]{0.56,0.35,0.01}{\textbf{\textit{#1}}}}
\newcommand{\AlertTok}[1]{\textcolor[rgb]{0.94,0.16,0.16}{#1}}
\newcommand{\ErrorTok}[1]{\textcolor[rgb]{0.64,0.00,0.00}{\textbf{#1}}}
\newcommand{\NormalTok}[1]{#1}
\usepackage{graphicx,grffile}
\makeatletter
\def\maxwidth{\ifdim\Gin@nat@width>\linewidth\linewidth\else\Gin@nat@width\fi}
\def\maxheight{\ifdim\Gin@nat@height>\textheight\textheight\else\Gin@nat@height\fi}
\makeatother
% Scale images if necessary, so that they will not overflow the page
% margins by default, and it is still possible to overwrite the defaults
% using explicit options in \includegraphics[width, height, ...]{}
\setkeys{Gin}{width=\maxwidth,height=\maxheight,keepaspectratio}
\IfFileExists{parskip.sty}{%
\usepackage{parskip}
}{% else
\setlength{\parindent}{0pt}
\setlength{\parskip}{6pt plus 2pt minus 1pt}
}
\setlength{\emergencystretch}{3em}  % prevent overfull lines
\providecommand{\tightlist}{%
  \setlength{\itemsep}{0pt}\setlength{\parskip}{0pt}}
\setcounter{secnumdepth}{0}
% Redefines (sub)paragraphs to behave more like sections
\ifx\paragraph\undefined\else
\let\oldparagraph\paragraph
\renewcommand{\paragraph}[1]{\oldparagraph{#1}\mbox{}}
\fi
\ifx\subparagraph\undefined\else
\let\oldsubparagraph\subparagraph
\renewcommand{\subparagraph}[1]{\oldsubparagraph{#1}\mbox{}}
\fi

%%% Use protect on footnotes to avoid problems with footnotes in titles
\let\rmarkdownfootnote\footnote%
\def\footnote{\protect\rmarkdownfootnote}

%%% Change title format to be more compact
\usepackage{titling}

% Create subtitle command for use in maketitle
\newcommand{\subtitle}[1]{
  \posttitle{
    \begin{center}\large#1\end{center}
    }
}

\setlength{\droptitle}{-2em}
  \title{Introduction to R}
  \pretitle{\vspace{\droptitle}\centering\huge}
  \posttitle{\par}
  \author{Abdirisak Mohamed}
  \preauthor{\centering\large\emph}
  \postauthor{\par}
  \predate{\centering\large\emph}
  \postdate{\par}
  \date{Spring 2018}


\begin{document}
\maketitle

{
\setcounter{tocdepth}{2}
\tableofcontents
}
\section{Basics of R}\label{basics-of-r}

RStudio is an integrated development environment for R. One of the
advantages of R is that you can start off as a user. It has an
interactive console where you can enter commands. It is similar to a
calculator.

\subsection{Arithmetic}\label{arithmetic}

\begin{Shaded}
\begin{Highlighting}[]
\OperatorTok{>}\StringTok{ }\DecValTok{4} \OperatorTok{+}\StringTok{ }\DecValTok{5}
\NormalTok{[}\DecValTok{1}\NormalTok{] }\DecValTok{9}
\end{Highlighting}
\end{Shaded}

\begin{Shaded}
\begin{Highlighting}[]
\OperatorTok{>}\StringTok{  }\DecValTok{1} \OperatorTok{+}\StringTok{ }\DecValTok{2}\OperatorTok{*}\DecValTok{3} \OperatorTok{-}\StringTok{ }\DecValTok{4}\OperatorTok{^}\DecValTok{2}
\NormalTok{[}\DecValTok{1}\NormalTok{] }\OperatorTok{-}\DecValTok{9}
\end{Highlighting}
\end{Shaded}

\begin{Shaded}
\begin{Highlighting}[]
\OperatorTok{>}\StringTok{  }\DecValTok{1} \OperatorTok{+}\StringTok{ }\NormalTok{(}\DecValTok{2}\OperatorTok{*}\DecValTok{3}\NormalTok{) }\OperatorTok{-}\StringTok{ }\NormalTok{(}\DecValTok{4}\OperatorTok{^}\DecValTok{2}\NormalTok{)}
\NormalTok{[}\DecValTok{1}\NormalTok{] }\OperatorTok{-}\DecValTok{9}
\end{Highlighting}
\end{Shaded}

\subsection{R Objects}\label{r-objects}

R has five atomic (basic) classes of objects:

\begin{itemize}
\tightlist
\item
  Character
\item
  Numeric (real numbers)
\item
  Integer
\item
  Complex
\item
  Logic (TRUE/FALSE)
\end{itemize}

\subsection{Examples (the \textless{}- is the assignment
operator)}\label{examples-the---is-the-assignment-operator}

\begin{Shaded}
\begin{Highlighting}[]
\OperatorTok{>}\StringTok{ }\NormalTok{name<-}\StringTok{"Rockville"}
\OperatorTok{>}\StringTok{ }\NormalTok{x<-}\KeywordTok{sqrt}\NormalTok{(}\DecValTok{2}\NormalTok{)}
\OperatorTok{>}\StringTok{ }\NormalTok{n<-8L}
\OperatorTok{>}\StringTok{ }\NormalTok{z<-}\DecValTok{3} \OperatorTok{+}\StringTok{ }\NormalTok{4i}
\OperatorTok{>}\StringTok{ }\NormalTok{statement<-}\OtherTok{FALSE}
\end{Highlighting}
\end{Shaded}

The \# character signals the start of a comment. \# and everything after
it are not evaluated

\begin{Shaded}
\begin{Highlighting}[]
\OperatorTok{>}\StringTok{ }\NormalTok{y<-}\DecValTok{9} \CommentTok{# not evaluated}
\end{Highlighting}
\end{Shaded}

We used the function sqrt() above. It is one of many built-in functions
in R.

\href{http://www.statmethods.net/management/functions.html}{Built-In
Functions}

\subsection{Print. You can either auto-print or use the function
print()}\label{print.-you-can-either-auto-print-or-use-the-function-print}

\begin{Shaded}
\begin{Highlighting}[]
\OperatorTok{>}\StringTok{ }\NormalTok{y}
\NormalTok{[}\DecValTok{1}\NormalTok{] }\DecValTok{9}
\OperatorTok{>}\StringTok{ }\KeywordTok{print}\NormalTok{(y)}
\NormalTok{[}\DecValTok{1}\NormalTok{] }\DecValTok{9}
\end{Highlighting}
\end{Shaded}

\subsection{Integer sequence}\label{integer-sequence}

\begin{Shaded}
\begin{Highlighting}[]
\OperatorTok{>}\StringTok{ }\NormalTok{z<-}\DecValTok{1}\OperatorTok{:}\DecValTok{10}
\OperatorTok{>}\StringTok{ }\NormalTok{z}
\NormalTok{ [}\DecValTok{1}\NormalTok{]  }\DecValTok{1}  \DecValTok{2}  \DecValTok{3}  \DecValTok{4}  \DecValTok{5}  \DecValTok{6}  \DecValTok{7}  \DecValTok{8}  \DecValTok{9} \DecValTok{10}
\OperatorTok{>}\StringTok{ }\NormalTok{x <-}\StringTok{ }\KeywordTok{seq}\NormalTok{(}\DataTypeTok{from=}\DecValTok{0}\NormalTok{, }\DataTypeTok{to=}\DecValTok{10}\NormalTok{, }\DataTypeTok{by=}\DecValTok{2}\NormalTok{)}
\OperatorTok{>}\StringTok{ }\NormalTok{x}
\NormalTok{[}\DecValTok{1}\NormalTok{]  }\DecValTok{0}  \DecValTok{2}  \DecValTok{4}  \DecValTok{6}  \DecValTok{8} \DecValTok{10}
\OperatorTok{>}\StringTok{ }\KeywordTok{length}\NormalTok{(x)}
\NormalTok{[}\DecValTok{1}\NormalTok{] }\DecValTok{6}
\end{Highlighting}
\end{Shaded}

\subsection{Vectors: The most basic
object}\label{vectors-the-most-basic-object}

A vector can only contain elements of the same class

\begin{Shaded}
\begin{Highlighting}[]
\OperatorTok{>}\StringTok{ }\NormalTok{w<-}\KeywordTok{c}\NormalTok{(}\StringTok{"name"}\NormalTok{, }\StringTok{"age"}\NormalTok{, }\StringTok{"gender"}\NormalTok{)}
\OperatorTok{>}\StringTok{ }\NormalTok{x<-}\KeywordTok{c}\NormalTok{(}\StringTok{"TRUE"}\NormalTok{, }\StringTok{"FALSE"}\NormalTok{)}
\OperatorTok{>}\StringTok{ }\NormalTok{y<-}\KeywordTok{c}\NormalTok{(}\FloatTok{1.5}\NormalTok{, }\FloatTok{2.7}\NormalTok{)}
\OperatorTok{>}\StringTok{ }\NormalTok{z<-}\DecValTok{1}\OperatorTok{:}\DecValTok{10}
\end{Highlighting}
\end{Shaded}

The function vector() creates a vector

\begin{Shaded}
\begin{Highlighting}[]
\OperatorTok{>}\StringTok{ }\NormalTok{x<-}\KeywordTok{vector}\NormalTok{(}\StringTok{"numeric"}\NormalTok{, }\DataTypeTok{length =} \DecValTok{4}\NormalTok{)}
\OperatorTok{>}\StringTok{ }\NormalTok{x}
\NormalTok{[}\DecValTok{1}\NormalTok{] }\DecValTok{0} \DecValTok{0} \DecValTok{0} \DecValTok{0}
\end{Highlighting}
\end{Shaded}

Index of a vector starts at 1 not 0

\begin{Shaded}
\begin{Highlighting}[]
\OperatorTok{>}\StringTok{ }\NormalTok{x<-}\KeywordTok{c}\NormalTok{(}\DecValTok{7}\NormalTok{,}\DecValTok{8}\NormalTok{,}\DecValTok{9}\NormalTok{)}
\OperatorTok{>}\StringTok{ }\NormalTok{x[}\DecValTok{1}\NormalTok{]}
\NormalTok{[}\DecValTok{1}\NormalTok{] }\DecValTok{7}
\OperatorTok{>}\StringTok{ }\NormalTok{x[}\DecValTok{0}\NormalTok{] }\CommentTok{# numeric vector of length zero (i.e., empty)}
\KeywordTok{numeric}\NormalTok{(}\DecValTok{0}\NormalTok{)}
\end{Highlighting}
\end{Shaded}

\subsection{Implicit Coersion:}\label{implicit-coersion}

If you put elements of different classes in a vector, R coerces them
into one class

\begin{Shaded}
\begin{Highlighting}[]
\OperatorTok{>}\StringTok{ }\NormalTok{x <-}\StringTok{ }\KeywordTok{c}\NormalTok{(}\FloatTok{2.9}\NormalTok{, }\StringTok{"b"}\NormalTok{)   ## character }
\OperatorTok{>}\StringTok{ }\KeywordTok{class}\NormalTok{(x)}
\NormalTok{[}\DecValTok{1}\NormalTok{] }\StringTok{"character"}
\OperatorTok{>}\StringTok{ }\NormalTok{y <-}\StringTok{ }\KeywordTok{c}\NormalTok{(}\DecValTok{5}\NormalTok{, }\OtherTok{FALSE}\NormalTok{)    ## numeric }
\OperatorTok{>}\StringTok{ }\KeywordTok{class}\NormalTok{(y)}
\NormalTok{[}\DecValTok{1}\NormalTok{] }\StringTok{"numeric"}
\OperatorTok{>}\StringTok{ }\NormalTok{z <-}\StringTok{ }\KeywordTok{c}\NormalTok{(}\StringTok{"b"}\NormalTok{, }\OtherTok{TRUE}\NormalTok{)  ## character}
\OperatorTok{>}\StringTok{ }\KeywordTok{class}\NormalTok{(z)}
\NormalTok{[}\DecValTok{1}\NormalTok{] }\StringTok{"character"}
\end{Highlighting}
\end{Shaded}

\subsection{Explicit Coersion}\label{explicit-coersion}

You can be explicitly coerce a vector from one class to another using
exisitng as.* functions

\begin{Shaded}
\begin{Highlighting}[]
\OperatorTok{>}\StringTok{ }\NormalTok{x <-}\StringTok{ }\DecValTok{1}\OperatorTok{:}\DecValTok{3}  
\OperatorTok{>}\StringTok{ }\NormalTok{x}
\NormalTok{[}\DecValTok{1}\NormalTok{] }\DecValTok{1} \DecValTok{2} \DecValTok{3}
\OperatorTok{>}\StringTok{ }\KeywordTok{class}\NormalTok{(x) }
\NormalTok{[}\DecValTok{1}\NormalTok{] }\StringTok{"integer"}
\OperatorTok{>}\StringTok{ }\KeywordTok{as.numeric}\NormalTok{(x) }
\NormalTok{[}\DecValTok{1}\NormalTok{] }\DecValTok{1} \DecValTok{2} \DecValTok{3}
\OperatorTok{>}\StringTok{ }\KeywordTok{as.logical}\NormalTok{(x)  }\CommentTok{# 0 = FALSE, other integers = TRUE}
\NormalTok{[}\DecValTok{1}\NormalTok{] }\OtherTok{TRUE} \OtherTok{TRUE} \OtherTok{TRUE}
\OperatorTok{>}\StringTok{ }\KeywordTok{as.character}\NormalTok{(x)}
\NormalTok{[}\DecValTok{1}\NormalTok{] }\StringTok{"1"} \StringTok{"2"} \StringTok{"3"}
\OperatorTok{>}\StringTok{ }
\ErrorTok{>}\StringTok{ }\CommentTok{# You get NAs as the conversion is not supported}
\ErrorTok{>}\StringTok{ }\NormalTok{y <-}\StringTok{ }\KeywordTok{c}\NormalTok{(}\StringTok{"a"}\NormalTok{, }\StringTok{"b"}\NormalTok{, }\StringTok{"c"}\NormalTok{) }
\OperatorTok{>}\StringTok{ }\NormalTok{y}
\NormalTok{[}\DecValTok{1}\NormalTok{] }\StringTok{"a"} \StringTok{"b"} \StringTok{"c"}
\OperatorTok{>}\StringTok{ }\KeywordTok{as.numeric}\NormalTok{(y) }
\NormalTok{Warning}\OperatorTok{:}\StringTok{ }\NormalTok{NAs introduced by coercion}
\NormalTok{[}\DecValTok{1}\NormalTok{] }\OtherTok{NA} \OtherTok{NA} \OtherTok{NA}
\OperatorTok{>}\StringTok{ }\KeywordTok{as.logical}\NormalTok{(y) }
\NormalTok{[}\DecValTok{1}\NormalTok{] }\OtherTok{NA} \OtherTok{NA} \OtherTok{NA}
\OperatorTok{>}\StringTok{ }
\ErrorTok{>}\StringTok{ }\CommentTok{# Logical to numeric}
\ErrorTok{>}\StringTok{ }\NormalTok{z<-}\KeywordTok{c}\NormalTok{(}\OtherTok{TRUE}\NormalTok{, }\OtherTok{FALSE}\NormalTok{, }\OtherTok{TRUE}\NormalTok{)}
\OperatorTok{>}\StringTok{ }\NormalTok{z}
\NormalTok{[}\DecValTok{1}\NormalTok{]  }\OtherTok{TRUE} \OtherTok{FALSE}  \OtherTok{TRUE}
\OperatorTok{>}\StringTok{ }\KeywordTok{as.numeric}\NormalTok{(z)}
\NormalTok{[}\DecValTok{1}\NormalTok{] }\DecValTok{1} \DecValTok{0} \DecValTok{1}
\OperatorTok{>}\StringTok{ }\KeywordTok{as.character}\NormalTok{(z)}
\NormalTok{[}\DecValTok{1}\NormalTok{] }\StringTok{"TRUE"}  \StringTok{"FALSE"} \StringTok{"TRUE"} 
\end{Highlighting}
\end{Shaded}

\subsection{Matrices}\label{matrices}

Matrices are vectors with dimensions. The dimension attribute is a
vector v = c(nrow, ncol) of integers. Like vecors, matrices contain
elements of the same class.

\begin{Shaded}
\begin{Highlighting}[]
\OperatorTok{>}\StringTok{ }\NormalTok{my_mat <-}\StringTok{ }\KeywordTok{matrix}\NormalTok{(}\DataTypeTok{nrow =} \DecValTok{3}\NormalTok{, }\DataTypeTok{ncol =} \DecValTok{4}\NormalTok{) }
\OperatorTok{>}\StringTok{ }\NormalTok{my_mat    }
\NormalTok{     [,}\DecValTok{1}\NormalTok{] [,}\DecValTok{2}\NormalTok{] [,}\DecValTok{3}\NormalTok{] [,}\DecValTok{4}\NormalTok{]}
\NormalTok{[}\DecValTok{1}\NormalTok{,]   }\OtherTok{NA}   \OtherTok{NA}   \OtherTok{NA}   \OtherTok{NA}
\NormalTok{[}\DecValTok{2}\NormalTok{,]   }\OtherTok{NA}   \OtherTok{NA}   \OtherTok{NA}   \OtherTok{NA}
\NormalTok{[}\DecValTok{3}\NormalTok{,]   }\OtherTok{NA}   \OtherTok{NA}   \OtherTok{NA}   \OtherTok{NA}
\OperatorTok{>}\StringTok{ }\KeywordTok{dim}\NormalTok{(my_mat)  }
\NormalTok{[}\DecValTok{1}\NormalTok{] }\DecValTok{3} \DecValTok{4}
\OperatorTok{>}\StringTok{ }\KeywordTok{attributes}\NormalTok{(my_mat) }
\OperatorTok{$}\NormalTok{dim}
\NormalTok{[}\DecValTok{1}\NormalTok{] }\DecValTok{3} \DecValTok{4}
\end{Highlighting}
\end{Shaded}

Matrices are filled column-wise. You can change it by setting byrow =
TRUE.

\begin{Shaded}
\begin{Highlighting}[]
\OperatorTok{>}\StringTok{ }\NormalTok{mat_}\DecValTok{2}\NormalTok{<-}\StringTok{ }\KeywordTok{matrix}\NormalTok{(}\DecValTok{1}\OperatorTok{:}\DecValTok{6}\NormalTok{, }\DataTypeTok{nrow =} \DecValTok{2}\NormalTok{, }\DataTypeTok{ncol =} \DecValTok{3}\NormalTok{) }
\OperatorTok{>}\StringTok{ }\NormalTok{mat_}\DecValTok{2}
\NormalTok{     [,}\DecValTok{1}\NormalTok{] [,}\DecValTok{2}\NormalTok{] [,}\DecValTok{3}\NormalTok{]}
\NormalTok{[}\DecValTok{1}\NormalTok{,]    }\DecValTok{1}    \DecValTok{3}    \DecValTok{5}
\NormalTok{[}\DecValTok{2}\NormalTok{,]    }\DecValTok{2}    \DecValTok{4}    \DecValTok{6}
\OperatorTok{>}\StringTok{ }
\ErrorTok{>}\StringTok{ }\CommentTok{# Fill by rows}
\ErrorTok{>}\StringTok{ }\NormalTok{mat_}\DecValTok{3}\NormalTok{<-}\StringTok{ }\KeywordTok{matrix}\NormalTok{(}\DecValTok{1}\OperatorTok{:}\DecValTok{6}\NormalTok{, }\DataTypeTok{nrow =} \DecValTok{2}\NormalTok{, }\DataTypeTok{ncol =} \DecValTok{3}\NormalTok{, }\DataTypeTok{byrow =} \OtherTok{TRUE}\NormalTok{) }
\OperatorTok{>}\StringTok{ }\NormalTok{mat_}\DecValTok{3}
\NormalTok{     [,}\DecValTok{1}\NormalTok{] [,}\DecValTok{2}\NormalTok{] [,}\DecValTok{3}\NormalTok{]}
\NormalTok{[}\DecValTok{1}\NormalTok{,]    }\DecValTok{1}    \DecValTok{2}    \DecValTok{3}
\NormalTok{[}\DecValTok{2}\NormalTok{,]    }\DecValTok{4}    \DecValTok{5}    \DecValTok{6}
\end{Highlighting}
\end{Shaded}

Adding a dimension attribute to a vector makes it a matrix

\begin{Shaded}
\begin{Highlighting}[]
\OperatorTok{>}\StringTok{ }\NormalTok{x<-}\DecValTok{1}\OperatorTok{:}\DecValTok{6}
\OperatorTok{>}\StringTok{ }\NormalTok{x}
\NormalTok{[}\DecValTok{1}\NormalTok{] }\DecValTok{1} \DecValTok{2} \DecValTok{3} \DecValTok{4} \DecValTok{5} \DecValTok{6}
\OperatorTok{>}\StringTok{ }\KeywordTok{dim}\NormalTok{(x)<-}\KeywordTok{c}\NormalTok{(}\DecValTok{2}\NormalTok{,}\DecValTok{3}\NormalTok{)}
\OperatorTok{>}\StringTok{ }\NormalTok{x}
\NormalTok{     [,}\DecValTok{1}\NormalTok{] [,}\DecValTok{2}\NormalTok{] [,}\DecValTok{3}\NormalTok{]}
\NormalTok{[}\DecValTok{1}\NormalTok{,]    }\DecValTok{1}    \DecValTok{3}    \DecValTok{5}
\NormalTok{[}\DecValTok{2}\NormalTok{,]    }\DecValTok{2}    \DecValTok{4}    \DecValTok{6}
\end{Highlighting}
\end{Shaded}

\section{Column-binding and
Row-binding}\label{column-binding-and-row-binding}

\begin{Shaded}
\begin{Highlighting}[]
\OperatorTok{>}\StringTok{ }\NormalTok{x<-}\DecValTok{1}\OperatorTok{:}\DecValTok{6}
\OperatorTok{>}\StringTok{ }\NormalTok{x}
\NormalTok{[}\DecValTok{1}\NormalTok{] }\DecValTok{1} \DecValTok{2} \DecValTok{3} \DecValTok{4} \DecValTok{5} \DecValTok{6}
\OperatorTok{>}\StringTok{ }\NormalTok{y<-}\DecValTok{11}\OperatorTok{:}\DecValTok{16}
\OperatorTok{>}\StringTok{ }\NormalTok{y}
\NormalTok{[}\DecValTok{1}\NormalTok{] }\DecValTok{11} \DecValTok{12} \DecValTok{13} \DecValTok{14} \DecValTok{15} \DecValTok{16}
\OperatorTok{>}\StringTok{ }\NormalTok{c_mat<-}\KeywordTok{cbind}\NormalTok{(x,y)}
\OperatorTok{>}\StringTok{ }\NormalTok{c_mat}
\NormalTok{     x  y}
\NormalTok{[}\DecValTok{1}\NormalTok{,] }\DecValTok{1} \DecValTok{11}
\NormalTok{[}\DecValTok{2}\NormalTok{,] }\DecValTok{2} \DecValTok{12}
\NormalTok{[}\DecValTok{3}\NormalTok{,] }\DecValTok{3} \DecValTok{13}
\NormalTok{[}\DecValTok{4}\NormalTok{,] }\DecValTok{4} \DecValTok{14}
\NormalTok{[}\DecValTok{5}\NormalTok{,] }\DecValTok{5} \DecValTok{15}
\NormalTok{[}\DecValTok{6}\NormalTok{,] }\DecValTok{6} \DecValTok{16}
\OperatorTok{>}\StringTok{ }\NormalTok{r_mat<-}\KeywordTok{rbind}\NormalTok{(x,y)}
\OperatorTok{>}\StringTok{ }\NormalTok{r_mat}
\NormalTok{  [,}\DecValTok{1}\NormalTok{] [,}\DecValTok{2}\NormalTok{] [,}\DecValTok{3}\NormalTok{] [,}\DecValTok{4}\NormalTok{] [,}\DecValTok{5}\NormalTok{] [,}\DecValTok{6}\NormalTok{]}
\NormalTok{x    }\DecValTok{1}    \DecValTok{2}    \DecValTok{3}    \DecValTok{4}    \DecValTok{5}    \DecValTok{6}
\NormalTok{y   }\DecValTok{11}   \DecValTok{12}   \DecValTok{13}   \DecValTok{14}   \DecValTok{15}   \DecValTok{16}
\end{Highlighting}
\end{Shaded}

\subsection{Lists}\label{lists}

Lists are like vectors but they can contain elements of different
classes

\begin{Shaded}
\begin{Highlighting}[]
\OperatorTok{>}\StringTok{ }\NormalTok{x <-}\StringTok{ }\KeywordTok{list}\NormalTok{(}\DecValTok{1}\OperatorTok{:}\DecValTok{6}\NormalTok{, }\KeywordTok{c}\NormalTok{(}\StringTok{"name"}\NormalTok{,}\StringTok{"gender"}\NormalTok{, }\StringTok{"age"}\NormalTok{), }\OtherTok{FALSE}\NormalTok{) }
\OperatorTok{>}\StringTok{ }\NormalTok{x}
\NormalTok{[[}\DecValTok{1}\NormalTok{]]}
\NormalTok{[}\DecValTok{1}\NormalTok{] }\DecValTok{1} \DecValTok{2} \DecValTok{3} \DecValTok{4} \DecValTok{5} \DecValTok{6}

\NormalTok{[[}\DecValTok{2}\NormalTok{]]}
\NormalTok{[}\DecValTok{1}\NormalTok{] }\StringTok{"name"}   \StringTok{"gender"} \StringTok{"age"}   

\NormalTok{[[}\DecValTok{3}\NormalTok{]]}
\NormalTok{[}\DecValTok{1}\NormalTok{] }\OtherTok{FALSE}
\end{Highlighting}
\end{Shaded}

Accessing elements of a list

\begin{Shaded}
\begin{Highlighting}[]
\OperatorTok{>}\StringTok{ }\NormalTok{x[[}\DecValTok{3}\NormalTok{]]}
\NormalTok{[}\DecValTok{1}\NormalTok{] }\OtherTok{FALSE}
\OperatorTok{>}\StringTok{ }\NormalTok{x[[}\DecValTok{2}\NormalTok{]][}\DecValTok{3}\NormalTok{]}
\NormalTok{[}\DecValTok{1}\NormalTok{] }\StringTok{"age"}
\OperatorTok{>}\StringTok{ }\NormalTok{x[[}\DecValTok{1}\NormalTok{]][}\DecValTok{5}\NormalTok{]}
\NormalTok{[}\DecValTok{1}\NormalTok{] }\DecValTok{5}
\end{Highlighting}
\end{Shaded}

\section{Factors}\label{factors}

Factors represent categorical variables

\begin{Shaded}
\begin{Highlighting}[]
\OperatorTok{>}\StringTok{ }\NormalTok{x <-}\StringTok{ }\KeywordTok{factor}\NormalTok{(}\KeywordTok{c}\NormalTok{(}\StringTok{"January"}\NormalTok{, }\StringTok{"April"}\NormalTok{, }\StringTok{"January"}\NormalTok{, }\StringTok{"June"}\NormalTok{, }\StringTok{"April"}\NormalTok{, }\StringTok{"April"}\NormalTok{))}
\OperatorTok{>}\StringTok{ }\NormalTok{x}
\NormalTok{[}\DecValTok{1}\NormalTok{] January April   January June    April   April  }
\NormalTok{Levels}\OperatorTok{:}\StringTok{ }\NormalTok{April January June}
\OperatorTok{>}\StringTok{ }\KeywordTok{table}\NormalTok{(x)}
\NormalTok{x}
\NormalTok{  April January    June }
      \DecValTok{3}       \DecValTok{2}       \DecValTok{1} 
\OperatorTok{>}\StringTok{ }\KeywordTok{unclass}\NormalTok{(x)}
\NormalTok{[}\DecValTok{1}\NormalTok{] }\DecValTok{2} \DecValTok{1} \DecValTok{2} \DecValTok{3} \DecValTok{1} \DecValTok{1}
\KeywordTok{attr}\NormalTok{(,}\StringTok{"levels"}\NormalTok{)}
\NormalTok{[}\DecValTok{1}\NormalTok{] }\StringTok{"April"}   \StringTok{"January"} \StringTok{"June"}   
\OperatorTok{>}\StringTok{ }\CommentTok{# You can set the order of the levels}
\ErrorTok{>}\StringTok{ }\NormalTok{y <-}\StringTok{ }\KeywordTok{factor}\NormalTok{(}\KeywordTok{c}\NormalTok{(}\StringTok{"January"}\NormalTok{, }\StringTok{"April"}\NormalTok{, }\StringTok{"January"}\NormalTok{, }\StringTok{"June"}\NormalTok{, }\StringTok{"April"}\NormalTok{, }\StringTok{"April"}\NormalTok{), }
\OperatorTok{+}\StringTok{             }\DataTypeTok{levels =} \KeywordTok{c}\NormalTok{(}\StringTok{"January"}\NormalTok{, }\StringTok{"April"}\NormalTok{, }\StringTok{"June"}\NormalTok{))}
\OperatorTok{>}\StringTok{ }\NormalTok{y}
\NormalTok{[}\DecValTok{1}\NormalTok{] January April   January June    April   April  }
\NormalTok{Levels}\OperatorTok{:}\StringTok{ }\NormalTok{January April June}
\end{Highlighting}
\end{Shaded}

\subsection{Missing Values}\label{missing-values}

NA stands for Not Available and NaN stands for Not a Number.

The function is.na() is used to test if an element or object is NA.

Similarly, is.nan() is for testing NaN.

NA values have a class,e.g., integer NA, character NA.

Every NaN value is also NA value but the converse does not hold

\begin{Shaded}
\begin{Highlighting}[]
\OperatorTok{>}\StringTok{  }\NormalTok{x<-}\KeywordTok{c}\NormalTok{(}\DecValTok{3}\NormalTok{, }\DecValTok{5}\NormalTok{, }\OtherTok{NA}\NormalTok{, }\StringTok{"a"}\NormalTok{, }\OtherTok{TRUE}\NormalTok{) }
\OperatorTok{>}\StringTok{ }\KeywordTok{is.na}\NormalTok{(x) }
\NormalTok{[}\DecValTok{1}\NormalTok{] }\OtherTok{FALSE} \OtherTok{FALSE}  \OtherTok{TRUE} \OtherTok{FALSE} \OtherTok{FALSE}
\OperatorTok{>}\StringTok{ }\KeywordTok{is.nan}\NormalTok{(x) }
\NormalTok{[}\DecValTok{1}\NormalTok{] }\OtherTok{FALSE} \OtherTok{FALSE} \OtherTok{FALSE} \OtherTok{FALSE} \OtherTok{FALSE}
\OperatorTok{>}\StringTok{ }\NormalTok{y<-}\KeywordTok{c}\NormalTok{(}\DecValTok{3}\NormalTok{, }\DecValTok{5}\NormalTok{, }\OtherTok{NA}\NormalTok{, }\StringTok{"a"}\NormalTok{, }\OtherTok{NaN}\NormalTok{) }
\OperatorTok{>}\StringTok{ }\KeywordTok{is.na}\NormalTok{(y) }
\NormalTok{[}\DecValTok{1}\NormalTok{] }\OtherTok{FALSE} \OtherTok{FALSE}  \OtherTok{TRUE} \OtherTok{FALSE} \OtherTok{FALSE}
\OperatorTok{>}\StringTok{ }\KeywordTok{is.nan}\NormalTok{(y) }
\NormalTok{[}\DecValTok{1}\NormalTok{] }\OtherTok{FALSE} \OtherTok{FALSE} \OtherTok{FALSE} \OtherTok{FALSE} \OtherTok{FALSE}
\end{Highlighting}
\end{Shaded}

\subsection{Data Frames}\label{data-frames}

Data Frames contain data in tabular form.

Data Frames are a special type of list; every member of the list has the
same length.

The columns are the members of the list.

Data frames can (in contrast to matrices) contain different classes of
objects

Data frames have the attribute row.names

If you want to read a file as a data frame, use read.table() or
read.csv().

Data Frames can be converted to a matrix by the function data.matrix()

\begin{Shaded}
\begin{Highlighting}[]
\OperatorTok{>}\StringTok{ }\NormalTok{df <-}\StringTok{ }\KeywordTok{data.frame}\NormalTok{( }\DataTypeTok{age =} \KeywordTok{c}\NormalTok{(}\DecValTok{20}\NormalTok{,}\DecValTok{30}\NormalTok{,}\DecValTok{25}\NormalTok{, }\DecValTok{23}\NormalTok{), }
\OperatorTok{+}\StringTok{                   }\DataTypeTok{class =} \KeywordTok{c}\NormalTok{(}\StringTok{"Freshman"}\NormalTok{, }\StringTok{"Senior"}\NormalTok{ , }\StringTok{"Junior"}\NormalTok{, }\StringTok{"Sophomore"}\NormalTok{) )}
\OperatorTok{>}\StringTok{ }\NormalTok{df}
\NormalTok{  age     class}
\DecValTok{1}  \DecValTok{20}\NormalTok{  Freshman}
\DecValTok{2}  \DecValTok{30}\NormalTok{    Senior}
\DecValTok{3}  \DecValTok{25}\NormalTok{    Junior}
\DecValTok{4}  \DecValTok{23}\NormalTok{ Sophomore}
\OperatorTok{>}\StringTok{ }\KeywordTok{nrow}\NormalTok{(df) }
\NormalTok{[}\DecValTok{1}\NormalTok{] }\DecValTok{4}
\OperatorTok{>}\StringTok{ }\KeywordTok{ncol}\NormalTok{(df) }
\NormalTok{[}\DecValTok{1}\NormalTok{] }\DecValTok{2}
\end{Highlighting}
\end{Shaded}

\subsection{Names}\label{names}

Vectors can have names

\begin{Shaded}
\begin{Highlighting}[]
\OperatorTok{>}\StringTok{ }\NormalTok{vec<-}\KeywordTok{c}\NormalTok{(}\DecValTok{25}\NormalTok{, }\FloatTok{5.5}\NormalTok{, }\DecValTok{150}\NormalTok{)}
\OperatorTok{>}\StringTok{ }\NormalTok{vec}
\NormalTok{[}\DecValTok{1}\NormalTok{]  }\FloatTok{25.0}   \FloatTok{5.5} \FloatTok{150.0}
\OperatorTok{>}\StringTok{ }\KeywordTok{names}\NormalTok{(vec)}
\OtherTok{NULL}
\OperatorTok{>}\StringTok{ }\KeywordTok{names}\NormalTok{(vec)<-}\KeywordTok{c}\NormalTok{(}\StringTok{"age"}\NormalTok{, }\StringTok{"height"}\NormalTok{, }\StringTok{"weight"}\NormalTok{)}
\OperatorTok{>}\StringTok{ }\NormalTok{vec}
\NormalTok{   age height weight }
  \FloatTok{25.0}    \FloatTok{5.5}  \FloatTok{150.0} 
\OperatorTok{>}\StringTok{ }\KeywordTok{names}\NormalTok{(vec)}
\NormalTok{[}\DecValTok{1}\NormalTok{] }\StringTok{"age"}    \StringTok{"height"} \StringTok{"weight"}
\end{Highlighting}
\end{Shaded}

Lists with names

\begin{Shaded}
\begin{Highlighting}[]
\OperatorTok{>}\StringTok{ }\NormalTok{z<-}\KeywordTok{list}\NormalTok{(}\DataTypeTok{age =} \KeywordTok{c}\NormalTok{(}\DecValTok{20}\NormalTok{,}\DecValTok{30}\NormalTok{,}\DecValTok{25}\NormalTok{, }\DecValTok{23}\NormalTok{), }\DataTypeTok{class =} \KeywordTok{c}\NormalTok{(}\StringTok{"Freshman"}\NormalTok{, }\StringTok{"Senior"}\NormalTok{ , }\StringTok{"Junior"}\NormalTok{, }\StringTok{"Sophomore"}\NormalTok{))}
\OperatorTok{>}\StringTok{ }\NormalTok{z}
\OperatorTok{$}\NormalTok{age}
\NormalTok{[}\DecValTok{1}\NormalTok{] }\DecValTok{20} \DecValTok{30} \DecValTok{25} \DecValTok{23}

\OperatorTok{$}\NormalTok{class}
\NormalTok{[}\DecValTok{1}\NormalTok{] }\StringTok{"Freshman"}  \StringTok{"Senior"}    \StringTok{"Junior"}    \StringTok{"Sophomore"}
\end{Highlighting}
\end{Shaded}

Accessing a list by names

\begin{Shaded}
\begin{Highlighting}[]
\OperatorTok{>}\StringTok{ }\NormalTok{z}\OperatorTok{$}\NormalTok{age}
\NormalTok{[}\DecValTok{1}\NormalTok{] }\DecValTok{20} \DecValTok{30} \DecValTok{25} \DecValTok{23}
\OperatorTok{>}\StringTok{ }\NormalTok{z}\OperatorTok{$}\NormalTok{class}
\NormalTok{[}\DecValTok{1}\NormalTok{] }\StringTok{"Freshman"}  \StringTok{"Senior"}    \StringTok{"Junior"}    \StringTok{"Sophomore"}
\end{Highlighting}
\end{Shaded}

Matrices with names

\begin{Shaded}
\begin{Highlighting}[]
\OperatorTok{>}\StringTok{ }\NormalTok{my_mat <-}\StringTok{ }\KeywordTok{matrix}\NormalTok{(}\DecValTok{1}\OperatorTok{:}\DecValTok{6}\NormalTok{, }\DataTypeTok{nrow =} \DecValTok{2}\NormalTok{, }\DataTypeTok{ncol =} \DecValTok{3}\NormalTok{) }
\OperatorTok{>}\StringTok{ }\KeywordTok{dimnames}\NormalTok{(my_mat) <-}\StringTok{ }\KeywordTok{list}\NormalTok{(}\KeywordTok{c}\NormalTok{(}\StringTok{"a"}\NormalTok{, }\StringTok{"b"}\NormalTok{), }\KeywordTok{c}\NormalTok{(}\StringTok{"c"}\NormalTok{, }\StringTok{"d"}\NormalTok{, }\StringTok{"e"}\NormalTok{)) }
\OperatorTok{>}\StringTok{ }\NormalTok{my_mat}
\NormalTok{  c d e}
\NormalTok{a }\DecValTok{1} \DecValTok{3} \DecValTok{5}
\NormalTok{b }\DecValTok{2} \DecValTok{4} \DecValTok{6}
\end{Highlighting}
\end{Shaded}

Accessing names

\begin{Shaded}
\begin{Highlighting}[]
\OperatorTok{>}\StringTok{ }\KeywordTok{names}\NormalTok{(df)}
\NormalTok{[}\DecValTok{1}\NormalTok{] }\StringTok{"age"}   \StringTok{"class"}
\OperatorTok{>}\StringTok{ }\KeywordTok{colnames}\NormalTok{(df)}
\NormalTok{[}\DecValTok{1}\NormalTok{] }\StringTok{"age"}   \StringTok{"class"}
\OperatorTok{>}\StringTok{ }\KeywordTok{rownames}\NormalTok{(df)}
\NormalTok{[}\DecValTok{1}\NormalTok{] }\StringTok{"1"} \StringTok{"2"} \StringTok{"3"} \StringTok{"4"}
\OperatorTok{>}\StringTok{ }\KeywordTok{row.names}\NormalTok{(df)}
\NormalTok{[}\DecValTok{1}\NormalTok{] }\StringTok{"1"} \StringTok{"2"} \StringTok{"3"} \StringTok{"4"}
\end{Highlighting}
\end{Shaded}


\end{document}
